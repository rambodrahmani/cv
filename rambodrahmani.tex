%-------------------------------------------------------------------------------
% File: rambodrahmani.tex
%       Rambod Rahmani CV Thesis template.
%
%       Compile using:
%           $ lualatex rambodrahmani.tex
%
% Author: Rambod Rahmani <rambodrahmani@autistici.org>
%         Created on 28/09/2017
%-------------------------------------------------------------------------------

% Add 'print' as an option into the square bracket to remove colors from this
% template for printing
\documentclass[]{friggeri-cv}

\usepackage{changepage}
\usepackage{csquotes}
\hypersetup{colorlinks=true,urlcolor={blue}}
\def\UrlBreaks{\do\/\do-}

\begin{document}

 % Name and current job title/field
\header{Rambod}{Rahmani}{Artificial Intelligence Engineer}

%-------------------------------------------------------------------------------
% SIDEBAR SECTION 1
%-------------------------------------------------------------------------------

% In the aside, each new line forces a line break
\begin{aside}
\section{Contact}
Via Pietro Casu, 16
Sassari, (SS) 07100
Italy
~
(+39) 333 3840979
(+39) 079 2859049
~
\small{{\faEnvelope}~{\addfontfeature{Color=blue}rrahmani@rhimpianti.it}}
\small{{\faGlobe}~{\addfontfeature{Color=blue}rambodrahmani.it}}\vspace{6mm}
\section{Languages}
Persian (Mother tongue)
Italian (Mother tongue)
English (C1)
French (A2)
Spanish (A1)\vspace{6mm}
\section{Places \textcolor{light-blue}{\faMapMarker}}
Iran, Italy, Germany, Spain, France, Belgium, Holland, Russia.\vspace{6mm}
\section{Programming \small{\textcolor{magenta}{{\faHeart}}}}
C, C++, Java, JavaFX, Python, Objective-C, .NET Framework, R, Latex, OpenCV, OpenCL, OpenGL, Nvidia CUDA, APEX, SQL, PL/SQL, MySQL, PHP, CSS, JavaScript, jQuery, HTML5, XML, JSON, Shell Script.\vspace{6mm}
\section{Software}
Vim, xCode, Android Studio, Windows Phone SDK, Matlab, Visual Studio, Eclipse, OMNeT++, VirtualBox, ISPConfig, MongoDB, CassandraDB, Metasploit, Wireshark, Git, R Studio, Jupyter Notebook, OpenMediaVault.\vspace{6mm}
\end{aside}

%-------------------------------------------------------------------------------
% SUMMARY SECTION
%-------------------------------------------------------------------------------

\section{Summary}
\vspace{-3mm}
I was born in Iran in 1993 and moved to Italy at the age of 7. I started
studying programming languages at the age of 15 and began to work as a freelance
programmer a couple of years later. At the age of 19 I was admitted to the
University of Pisa as a Computer Engineering student. After finishing my
bachelor I enrolled for a Master in Artificial Intelligence and Data
Engineering. In the past years, beside studying for my examinations, I worked
for IT Companies and for the Italian government, I took part in group projects
and tried to startup my own ideas. During the academic year 2017/2018, my goals
have shifted. I have developed a deep interest in Artificial Intelligence.

%-------------------------------------------------------------------------------
% EDUCATION SECTION
%-------------------------------------------------------------------------------

\section{Education}
\vspace{-3mm}
\begin{entrylist}
\entry
{2020 - 2022}
{Master {\normalfont\small{of Engineering Studies - Artificial Intelligence}}}
{University of Pisa, Italy}
{\vspace{-3mm}}
%------------------------------------------------
\entry
{2013 - 2019}
{Bachelor {\normalfont\small{of Engineering Studies - Computer Engineering}}}
{University of Pisa, Italy}
{\vspace{-3mm}}
%------------------------------------------------
\entry
{2008 - 2013}
{High School Diploma {\normalfont\small{100/100 - Scientific Lyceum}}}
{Scientific Lyceum "G. Spano", Italy}
{\vspace{-5mm}}
%------------------------------------------------
\end{entrylist}

%-------------------------------------------------------------------------------
% UNIVERSITY PROJECTS SECTION
%-------------------------------------------------------------------------------
\patchcmd{\entry}{14.8cm}{11.8cm}{}{}
\section{University Projects}
\vspace{-3mm}
The following is a list of university projects which I took part into and of
which I am a part during my Bachelor of Engineering and my Master of Engineering
studies.

\begin{entrylist}
%------------------------------------------------
\entry
{2020}
{E-Team Squadra Corse}
{University of Pisa}
{\emph{Driverless Autonomous Vehicle Development}\\
I am currently part of the University of Pisa Formula SAE team. My work focuses
on the development of the driverless autonomous vehicle.}
%------------------------------------------------
\entry
{2019}
{Roborace}
{University of Pisa - E. Piaggio Research Center}
{\emph{Autonomous Vehicle Development}\\
I was part of the Roborace team of the E. Piaggio Research Center during
Roborace championship Season Alpha. My work focused on the driverless electric
vehicle DevBot 2.0.}
%------------------------------------------------
\entry
{2015}
{PHOS Project}
{University of Pisa}
{\emph{Software Programmer}\\
iOS and Android App development for PHOS Project (Aerospace project at
University of Pisa part of the REXUS/BEXUS programme).}
%------------------------------------------------
\end{entrylist}

%-------------------------------------------------------------------------------
% WORK EXPERIENCE SECTION
%-------------------------------------------------------------------------------
\patchcmd{\entry}{14.8cm}{11.8cm}{}{}
\section{Work Experience}
\vspace{-3mm}
\begin{entrylist}
%------------------------------------------------
\entry
{2020}
{R.H. Impianti di Rahmani Hamidreza}
{Sassari}
{\emph{Chief Technical Officer (CTO)}.}
%------------------------------------------------
\entry
{2020}
{R.H. Impianti di Rahmani Hamidreza}
{Consorzio di Bonifica della Nurra}
{\emph{Siemens TIA Portal V16 Developer}\\
Impianto di Sollevamento di Maristella, di Donna Ricca e di Brunestica del
Consorzio di Bonifica della Nurra: progettazione, realizzazione e programmazione
del sistema di Telegestione basato su PLC Siemens S7-1200, Simatic HMI TP1200
Comfort, Convertitore di frequenza SINAMICS G120X, Misuratore di energia Sentron
PAC3220, Interruttori SENTRON 3VA2, moduli SITOP DC UPS e Siemens SCALANCE X208.}
%------------------------------------------------
\end{entrylist}
\clearpage

%-------------------------------------------------------------------------------
% SIDEBAR SECTION 2
%-------------------------------------------------------------------------------
\begin{aside}\vspace{-2.8cm}
\section{Hardware}
Arduino, Raspberry Pi, Embedded Pi, Cubieboard, NVIDIA Jetson, Siemens PLC S7-1200, Simatic HMI TP1200 Comfort, SINAMICS G120X, Sentron PAC3220, SENTRON 3VA2, SITOP DC UPS, Siemens SCALANCE X208.
\section{Operating Systems \textcolor{orange}{\faLinux}}
Windows 10, Mac OS X, iOS, Android, Linux Debian, Arch Linux, Raspbian, OpenWRT,
Kali Linux, Parrot OS, Tails.\vspace{6mm}
\section{Linkedin \textcolor{blue}{\faLinkedinSquare}}
For full and detailed information regarding my work experience, projects and
awards please refer to my \textbf{Linkedin Profile}:
\href{http://it.linkedin.com/in/rambodrahmani}{{\addfontfeature{Color=blue}it.linkedin.com/in/}}\\\href{http://it.linkedin.com/in/rambodrahmani}{{\addfontfeature{Color=blue}rambodrahmani}}.\vspace{6mm}
\section{GitHub \textcolor{teal}{\faGithub}}
\emph{"Sharing isn't immoral - it's a moral imperative. Only those blinded by
greed would refuse to let a friend make a copy."}\vspace{1mm}
\textbf{GitHub Profile}: \href{http://github.com/rambodrahmani}{{\addfontfeature{Color=blue}github.com/}}\\\href{http://github.com/rambodrahmani}{{\addfontfeature{Color=blue}rambodrahmani}}.\vspace{6mm}
\end{aside}

%-------------------------------------------------------------------------------
% WORK EXPERIENCE SECTION
%-------------------------------------------------------------------------------
\begin{entrylist}
%------------------------------------------------
\entry
{2018}
{ClassX S.r.l.}
{Pontedera}
{\emph{Java Developer} \\
I worked as Java developer for the ClassX S.r.l focusing to develop broadcasting software and hardware. I got a glimpse of what it means to work with Modern OpenGL, writing native drivers for custom broadcasting hardware as well as using JNI.}
%------------------------------------------------
\entry
{2018}
{ClassX S.r.l.}
{Pontedera}
{\emph{Intership} \\
I spent two months at ClassX S.r.l as part of my academic career as a Computer Science student at University of Pisa. I developed an Android Application used to receive SMS messages and forward them to the company software for broadcasting purposes.}
%------------------------------------------------
\entry
{2014 - 2015}
{Boccaccio Club}
{Pisa, Italy}
{\emph{Software Programmer} \\
Boccaccio Club iOS and Android App Development. Wireless Network Routing with Multiple Access Points: internet access is guaranteed only when the Boccaccio Club App is installed on your device. In order to achieve such an advanced authentication method I developed a custom captive portal named DYNAMO.}
%------------------------------------------------
\entry
{2013}
{Ciclope: Raspberry Pi based Humanoid}
{Rambod Rahmani, Giacomo Taormina}
{\emph{Software Development \& Hardware Assembly} \\
Ciclope is a Humanoid robot with the ability to see and analyse the external world using a camera. Its structure has up to 17 degrees of freedom.\\
It is a one-of-a-kind combination of Software and Hardware: it combines the latest OpenCV libraries and the RPi on board camera. I coded basic functionalities such as object-tracking and face-recognition, and its control interface using Python, C and C++.}
%------------------------------------------------
\entry
{2013}
{Ambulatory Proctology Office}
{Sardinia, Italy}
{\emph{Software Programmer} \\
I developed a software named "Anamnesi" (Objective-C - MAC OS X) used to manage patients medical histories. Each patient is identified using his fiscal code and the ambulatory doctors are able to access all medical histories stored on a server wherever they are.}
%------------------------------------------------
\entry
{2012}
{MALIBU Village}
{Canet-Plage, Languedoc-Roussillon, France}
{\emph{Internship} \\
I worked as IT Support at MALIBU Village as part of the "Leonardo Project" internship. This was the first time I had a hands-on experience with Ubiquiti Picostations.}
%------------------------------------------------
\entry
{2010 - 2012}
{Italian Navy}
{San Giorgio-class amphibious transport dock, Italy}
{\emph{Software Programmer} \\
I coded a set of three software named \textbf{A.S.C (Acceptance, Sorting and Control)} for the Italian Navy. What started as a late night project to improve my coding skills ended up changing my life for ever when I was still a high school student. During more than two years of daily work I translated a NATO publication made in 2008 into machine language. Thanks to A.S.C. operators do not have to worry anymore about formatting and standardizing their messages: the software checks every line before sending the message and tells the operator how to correct any mistake.}
%------------------------------------------------
\end{entrylist}
\newpage

%----------------------------------------------------------------------------------------
%	SIDEBAR SECTION 3
%----------------------------------------------------------------------------------------

\begin{aside}\vspace{-2.8cm}
\section{3D Printing \textcolor{red}{\faConnectdevelop}}
\begin{itemize}
	\item Geeetech Prusa i3 Pro B
	\item Anycubic I3 Mega
	\item Creality Ender 3 Pro
	\item Anycubic Photon
\end{itemize}\vspace{6mm}
\section{OpenWRT \textcolor{green}{\faWifi}}
As a member of EigenLab, a local hacklab in Pisa, I was involved in installing and configuring the OpenWRT Distribution on embedded devices that we used to build EigenNet: a local \textbf{Wireless Mesh Network} in the city of Pisa.\vspace{6mm}
\section{Bitcoin \textcolor{red}{\faBitcoin}}
I have been involved in Bitcoins as far as it regards Bitcoin Mining (CGMiner, CPUMiner, USB Miners), Mining Pools, Bitcoin Wallet and Full Bitcoin Node Configuration (Bitcoin Core).\vspace{6mm}
\small{Last updated: \today\ }
\end{aside}

%----------------------------------------------------------------------------------------
%	AWARDS SECTION
%----------------------------------------------------------------------------------------

\section{Awards, Scholarships and Certificates}
\vspace{-3mm}
\begin{entrylist}
%------------------------------------------------
\entry
{2013 - 2016}
{DSU Scholarship}
{DSU of Tuscany}
{DSU Scholarship by DSU of Tuscany supporting my Bachelor of Engineering Studies.}
%------------------------------------------------
\entry
{2013 - 2016}
{Italian Ministry of Education, University and Research Scholarship}
{MIUR}
{Excellence Scholarship by MIUR supporting my Bachelor of Engineering Studies.}
%------------------------------------------------
\entry
{2012}
{Intel Excellence in Computer Science}
{Intel Corporation}
{Awarded to Rambod Rahmani for developing the A.S.C. Software.}
%------------------------------------------------
\entry
{2012}
{National Excellence Registry}
{Italian Ministry of Education, University and Research}
{Awarded to Rambod Rahmani for developing the A.S.C. Software.}
%------------------------------------------------
\entry
{2012}
{Certificate of Merit}
{Scientific Lyceum "G. Spano", Italy}
{Awarded to Rambod Rahmani for developing the A.S.C. Software.}
%------------------------------------------------
\entry
{2012}
{Europass Mobility Nr. IT/00/2012/859/10}
{}
{Gymnasium Lyceum "D. A. Azuni" and ALFMED (Acad\'emie de Langues France M\'editerran\'ee.).}
%------------------------------------------------
\entry
{2011}
{A.S.C. Software}
{Italian Navy - San Giorgio-class Amphibious Transport Dock}
{Software Development Certificate by the Italian Navy for developing the A.S.C. Software.}
%------------------------------------------------
\end{entrylist}

%----------------------------------------------------------------------------------------
%	PATENTS SECTION
%----------------------------------------------------------------------------------------

\section{Patents}
\vspace{-3mm}
\begin{entrylist}
%------------------------------------------------
\entry
{2015}
{DYNAMO}
{Ministry of Economic Development (Italy) - Request number: 102015000072776}
{Captive portal software which allows for Wi-Fi Access Points to be configured so as to provide internet access only to those devices where a required iOS/Android App is installed. Otherwise the device only has internet access as far as it is needed to install the required App. I developed this custom Captive Portal in order to achieve the advanced authentication method required by Boccaccio Club and thanks to their financial support I was able to patent DYNAMO.}
%------------------------------------------------
\end{entrylist}

%----------------------------------------------------------------------------------------
%	EVENTS SECTION
%----------------------------------------------------------------------------------------

\section{Events}
\vspace{-3mm}
\begin{entrylist}
%------------------------------------------------
\entry
{May 2015}
{International World Wide Web Conference 2015}
{Florence, Italy}
{Research, development, and applications of the topics related to the Web.}
%------------------------------------------------
\entry
{May 2015}
{HEUREKA 2015}
{Berlin, Germany}
{One of the main founders conferences in the German­-speaking world.}
%------------------------------------------------
\entry
{May 2015}
{Berlin Talent Summit 2015}
{Berlin, Germany}
{Europe's leading recruiting conference for high-potentials.}
%------------------------------------------------
\entry
{July 2012}
{9th EXPO-SCIENCES EUROPE 2012 (ESE)}
{Tula, Russia}
{9th MILSET Expo-Sciences Europe (ESE 2012) held in Tula, Russia.}
%------------------------------------------------
\entry
{April 2012}
{I Giovani E Le Scienze 2012}
{Federation of Scientific and Technical Associations}
{The A.S.C. Software was selected as one of the best 30 in the whole country.}
%------------------------------------------------
\end{entrylist}

%----------------------------------------------------------------------------------------
%	INTERESTS SECTION
%----------------------------------------------------------------------------------------

\section{Interests}
\vspace{-3mm}
\textbf{Professional:} Startup, Computer Science, Geolocation Systems, Hardware, Smartphones, iOS, Android, Embedded systems, Web development, App development, Software design, Networking, Linux, Hacking, Penetration Testing, Computational Complexity, Artificial Intelligence, Artificial Neural Networks, Computer Vision.\\
\textbf{Personal:} Running, Chess, Cosmology, Astronomical observations, Quantum Computer Science, Physics and Math.

\clearpage

\end{document}
