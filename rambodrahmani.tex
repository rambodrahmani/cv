%-------------------------------------------------------------------------------
% File: rambodrahmani.tex
%       Rambod Rahmani CV Thesis template.
%
%       Compile using:
%           $ lualatex rambodrahmani.tex
%
% Author: Rambod Rahmani <rambodrahmani@autistici.org>
%         Created on 28/09/2017
%-------------------------------------------------------------------------------

% Add 'print' as an option into the square bracket to remove colors from this
% template for printing
\documentclass[]{friggeri-cv}

\usepackage{changepage}
\usepackage{csquotes}
\hypersetup{colorlinks=true,urlcolor={blue}}
\def\UrlBreaks{\do\/\do-}

\begin{document}

 % Name and current job title/field
\header{Rambod}{Rahmani}{Artificial Intelligence Engineer}

%-------------------------------------------------------------------------------
% SIDEBAR SECTION 1
%-------------------------------------------------------------------------------

% In the aside, each new line forces a line break
\begin{aside}
\vspace{-1.0cm}
\section{Contact}
Via Pietro Casu, 16
Sassari, (SS) 07100
Italy
~
(+39) 333 3840979
(+39) 079 2859049
~
\small{\hspace{-1.0cm}{\faEnvelope}~{\addfontfeature{Color=blue}rambodrahmani@autistici.org}}
\small{{\faGlobe}~{\addfontfeature{Color=blue}rambodrahmani.it}}\vspace{6mm}
\section{Languages}
Persian (Mother tongue)
Italian (Mother tongue)
English (C1)
French (A2)\vspace{6mm}
\section{Places \textcolor{light-blue}{\faMapMarker}}
Iran, Italy, Germany, Spain, France, Belgium, Holland, Russia, Hungary.\vspace{6mm}
\section{Programming \small{\textcolor{magenta}{{\faHeart}}}}
C, C++, Java, Python, R, OpenCV, OpenCL, OpenGL, Nvidia CUDA, Julia, ROS, Shell Script.\vspace{6mm}
\section{Software}
Vim, CLion, Matlab, Visual Studio, Eclipse, OMNeT++, VirtualBox, MongoDB, CassandraDB, Git, R Studio, Jupyter Notebook.\vspace{6mm}
\section{Hardware}
Arduino, Raspberry Pi, Embedded Pi, Cubieboard, NVIDIA Jetson.\vspace{6mm}
\section{Operating Systems \textcolor{orange}{\faLinux}}
Windows 10, Mac OS X, iOS, Android, Linux Debian, Arch Linux, Raspbian, OpenWRT.\vspace{6mm}
\end{aside}

%-------------------------------------------------------------------------------
% SUMMARY SECTION
%-------------------------------------------------------------------------------

\section{Summary}
\vspace{-3mm}
I was born in Iran in 1993 and moved to Italy at the age of 7. I started
studying programming languages at the age of 15 and began to work as a freelance
programmer a couple of years later. At the age of 19 I was admitted to the
University of Pisa as a Computer Engineering student. After finishing my
bachelor I enrolled for a Master in Artificial Intelligence and Data
Engineering. In the past years, beside studying for my examinations, I worked
for IT Companies, I took part in group projects and tried to startup my own ideas.
During the academic year 2017/2018, my goals have shifted. I have developed a
deep interest in Artificial Intelligence.

%-------------------------------------------------------------------------------
% EDUCATION SECTION
%-------------------------------------------------------------------------------

\section{Education}
\vspace{-3mm}
\begin{entrylist}
\entry
{2020 - 2022}
{Master {\normalfont\small{of Engineering Studies - Artificial Intelligence}}}
{University of Pisa, Italy}
{\vspace{-3mm}}
%------------------------------------------------
\entry
{2013 - 2019}
{Bachelor {\normalfont\small{of Engineering Studies - Computer Engineering}}}
{University of Pisa, Italy}
{\vspace{-3mm}}
%------------------------------------------------
\entry
{2008 - 2013}
{High School Diploma {\normalfont\small{100/100 - Scientific Lyceum}}}
{Scientific Lyceum "G. Spano", Italy}
{\vspace{-5mm}}
%------------------------------------------------
\end{entrylist}

%-------------------------------------------------------------------------------
% UNIVERSITY PROJECTS SECTION
%-------------------------------------------------------------------------------
\patchcmd{\entry}{14.8cm}{11.8cm}{}{}
\section{University Projects}
\vspace{-3mm}
\begin{entrylist}
%------------------------------------------------
\entry
{2020}
{E-Team Squadra Corse}
{University of Pisa}
{\emph{Autonomous Vehicle Development}\\
My work focused on the development of the driverless autonomous vehicle.}
%------------------------------------------------
\entry
{2019}
{Roborace}
{University of Pisa - E. Piaggio Research Center}
{\emph{Autonomous Vehicle Development}\\
My work focused on the driverless electric vehicle DevBot 2.0.}
%------------------------------------------------
\entry
{2015}
{PHOS Project}
{University of Pisa}
{\emph{Software Programmer}\\
Aerospace project at University of Pisa part of the REXUS/BEXUS programme.}
%------------------------------------------------
\end{entrylist}

%-------------------------------------------------------------------------------
% WORK EXPERIENCE SECTION
%-------------------------------------------------------------------------------
\patchcmd{\entry}{14.8cm}{11.8cm}{}{}
\section{Papers}
\vspace{-3mm}
\begin{entrylist}
%------------------------------------------------
\entry
{2012}
{KerubLess - Design of a Driverless Formula SAE Vehicle}
{IEEE}
{\emph{2022 IEEE 5th International Conference on Industrial Cyber-Physical Systems}}
%------------------------------------------------
\end{entrylist}

%-------------------------------------------------------------------------------
% WORK EXPERIENCE SECTION
%-------------------------------------------------------------------------------
\patchcmd{\entry}{14.8cm}{11.8cm}{}{}
\section{Work Experience}
\vspace{-3mm}
\begin{entrylist}
%------------------------------------------------
\entry
{2020}
{R.H. Impianti di Rahmani Hamidreza}
{Consorzio di Bonifica della Nurra}
{\emph{Siemens TIA Portal V16 Developer}\\
Design, development and deployment of a remote monitoring infastructure based
on Siemens PLC S7-1200, Simatic HMI TP1200, SINAMICS G120X.}
%------------------------------------------------
\entry
{2018}
{ClassX S.r.l.}
{Pontedera}
{\emph{Java Developer} \\
I worked as Java developer for the ClassX S.r.l focusing to develop broadcasting software and hardware.}
%------------------------------------------------
\entry
{2018}
{ClassX S.r.l.}
{Pontedera}
{\emph{Intership} \\
Curricular internship Computer Engineering student at University of Pisa.}
%------------------------------------------------
\entry
{2013}
{Ambulatory Proctology Office}
{Sardinia, Italy}
{\emph{Objective-C Software Programmer} \\
I developed the software used to manage patients medical history.}
%------------------------------------------------
\end{entrylist}
\clearpage

%-------------------------------------------------------------------------------
% SIDEBAR SECTION 2
%-------------------------------------------------------------------------------
\begin{aside}\vspace{-2.8cm}
\section{3D Printing \textcolor{red}{\faConnectdevelop}}
\begin{itemize}
	\item Geeetech Prusa i3
	\item Anycubic I3 Mega
	\item Creality Ender 3 Pro
	\item Anycubic Photon
\end{itemize}\vspace{6mm}
\section{OpenWRT \textcolor{green}{\faWifi}}
As a member of a local hacklab in Pisa, I was involved in installing and configuring the OpenWRT Distribution on embedded devices that we used to build the local \textbf{Wireless Mesh Network} EigenNet.\vspace{6mm}
\section{Bitcoin \textcolor{red}{\faBitcoin}}
I have been involved in Bitcoins as far as it regards Bitcoin Mining, Mining Pools, Bitcoin Wallet and Full Bitcoin Node Configuration (Bitcoin Core).\vspace{6mm}
\section{Linkedin \textcolor{blue}{\faLinkedinSquare}}
For full and detailed information regarding my work experience, projects and
awards please refer to my \textbf{Linkedin Profile}:
\href{http://it.linkedin.com/in/rambodrahmani}{{\addfontfeature{Color=blue}it.linkedin.com/in/}}\\\href{http://it.linkedin.com/in/rambodrahmani}{{\addfontfeature{Color=blue}rambodrahmani}}.\vspace{6mm}
\section{GitHub \textcolor{teal}{\faGithub}}
\emph{"Sharing isn't immoral - it's a moral imperative. Only those blinded by
greed would refuse to let a friend make a copy."}\vspace{1mm}
\textbf{GitHub Profile}: \href{http://github.com/rambodrahmani}{{\addfontfeature{Color=blue}github.com/}}\\\href{http://github.com/rambodrahmani}{{\addfontfeature{Color=blue}rambodrahmani}}.\vspace{10mm}
\small{Last updated: \today\ }
\end{aside}

%-------------------------------------------------------------------------------
% WORK EXPERIENCE SECTION
%-------------------------------------------------------------------------------
\begin{entrylist}
%------------------------------------------------
\entry
{2012}
{MALIBU Village}
{Canet-Plage, Languedoc-Roussillon, France}
{\emph{Internship} \\
I worked as IT Support at MALIBU Village as part of the "Leonardo Project" internship. I had a hands-on experience with Ubiquiti Picostations.}
%------------------------------------------------
\entry
{2010 - 2012}
{Italian Navy}
{San Giorgio-class amphibious transport dock, Italy}
{\emph{Software Programmer} \\
I coded a set of three software named \textbf{A.S.C (Acceptance, Sorting and Control)} for the Italian Navy.}
%------------------------------------------------
\entry
{2014 - 2015}
{Boccaccio Club}
{Pisa, Italy}
{\emph{Software Programmer} \\
Boccaccio Club iOS and Android App Development.}
%------------------------------------------------
\end{entrylist}

%----------------------------------------------------------------------------------------
%	AWARDS SECTION
%----------------------------------------------------------------------------------------

\section{Awards, Scholarships and Certificates}
\vspace{-3mm}
\begin{entrylist}
%------------------------------------------------
\entry
{2013 - 2016}
{DSU Scholarship}
{DSU of Tuscany}
{DSU Scholarship by DSU of Tuscany supporting my Bachelor of Engineering Studies.}
%------------------------------------------------
\entry
{2013 - 2016}
{Italian Ministry of Education, University and Research Scholarship}
{MIUR}
{Excellence Scholarship by MIUR supporting my Bachelor of Engineering Studies.}
%------------------------------------------------
\entry
{2012}
{Intel Excellence in Computer Science}
{Intel Corporation}
{Awarded to Rambod Rahmani for developing the A.S.C. Software.}
%------------------------------------------------
\entry
{2012}
{National Excellence Registry}
{Italian Ministry of Education, University and Research}
{Awarded to Rambod Rahmani for developing the A.S.C. Software.}
%------------------------------------------------
\entry
{2012}
{Certificate of Merit}
{Scientific Lyceum "G. Spano", Italy}
{Awarded to Rambod Rahmani for developing the A.S.C. Software.}
%------------------------------------------------
\entry
{2012}
{Europass Mobility Nr. IT/00/2012/859/10}
{}
{Gymnasium Lyceum "D. A. Azuni" and ALFMED (Acad\'emie de Langues France M\'editerran\'ee.).}
%------------------------------------------------
\entry
{2011}
{A.S.C. Software}
{Italian Navy - San Giorgio-class Amphibious Transport Dock}
{Software Development Certificate by the Italian Navy for developing the A.S.C. Software.}
%------------------------------------------------
\end{entrylist}

%----------------------------------------------------------------------------------------
%	EVENTS SECTION
%----------------------------------------------------------------------------------------

\section{Events}
\vspace{-3mm}
\begin{entrylist}
%------------------------------------------------
\entry
{May 2015}
{International World Wide Web Conference 2015}
{Florence, Italy}
{Research, development, and applications of the topics related to the Web.}
%------------------------------------------------
\entry
{May 2015}
{HEUREKA 2015}
{Berlin, Germany}
{One of the main founders conferences in the German­-speaking world.}
%------------------------------------------------
\entry
{May 2015}
{Berlin Talent Summit 2015}
{Berlin, Germany}
{Europe's leading recruiting conference for high-potentials.}
%------------------------------------------------
\entry
{July 2012}
{9th EXPO-SCIENCES EUROPE 2012 (ESE)}
{Tula, Russia}
{9th MILSET Expo-Sciences Europe (ESE 2012) held in Tula, Russia.}
%------------------------------------------------
\entry
{April 2012}
{I Giovani E Le Scienze 2012}
{Federation of Scientific and Technical Associations}
{The A.S.C. Software was selected as one of the best 30 in the whole country.}
%------------------------------------------------
\end{entrylist}

%----------------------------------------------------------------------------------------
%	INTERESTS SECTION
%----------------------------------------------------------------------------------------

\section{Interests}
\vspace{-3mm}
\small \textbf{Professional:} Computational Complexity, Artificial Intelligence, Artificial Neural Networks, Deep Learning, Symbolic Computation, Statistics, Hardware Accelerators, Cloud Computing, Software design, Linux.\\
\small \textbf{Personal:} Running, Chess, Cosmology, Astronomical observations, Stock Market, Quantitative trading.
\clearpage
\end{document}